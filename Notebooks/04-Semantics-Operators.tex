
% Default to the notebook output style

    


% Inherit from the specified cell style.




    
\documentclass[11pt]{article}

    
    
    \usepackage[T1]{fontenc}
    % Nicer default font than Computer Modern for most use cases
    \usepackage{palatino}

    % Basic figure setup, for now with no caption control since it's done
    % automatically by Pandoc (which extracts ![](path) syntax from Markdown).
    \usepackage{graphicx}
    % We will generate all images so they have a width \maxwidth. This means
    % that they will get their normal width if they fit onto the page, but
    % are scaled down if they would overflow the margins.
    \makeatletter
    \def\maxwidth{\ifdim\Gin@nat@width>\linewidth\linewidth
    \else\Gin@nat@width\fi}
    \makeatother
    \let\Oldincludegraphics\includegraphics
    % Set max figure width to be 80% of text width, for now hardcoded.
    \renewcommand{\includegraphics}[1]{\Oldincludegraphics[width=.8\maxwidth]{#1}}
    % Ensure that by default, figures have no caption (until we provide a
    % proper Figure object with a Caption API and a way to capture that
    % in the conversion process - todo).
    \usepackage{caption}
    \DeclareCaptionLabelFormat{nolabel}{}
    \captionsetup{labelformat=nolabel}

    \usepackage{adjustbox} % Used to constrain images to a maximum size 
    \usepackage{xcolor} % Allow colors to be defined
    \usepackage{enumerate} % Needed for markdown enumerations to work
    \usepackage{geometry} % Used to adjust the document margins
    \usepackage{amsmath} % Equations
    \usepackage{amssymb} % Equations
    \usepackage{textcomp} % defines textquotesingle
    % Hack from http://tex.stackexchange.com/a/47451/13684:
    \AtBeginDocument{%
        \def\PYZsq{\textquotesingle}% Upright quotes in Pygmentized code
    }
    \usepackage{upquote} % Upright quotes for verbatim code
    \usepackage{eurosym} % defines \euro
    \usepackage[mathletters]{ucs} % Extended unicode (utf-8) support
    \usepackage[utf8x]{inputenc} % Allow utf-8 characters in the tex document
    \usepackage{fancyvrb} % verbatim replacement that allows latex
    \usepackage{grffile} % extends the file name processing of package graphics 
                         % to support a larger range 
    % The hyperref package gives us a pdf with properly built
    % internal navigation ('pdf bookmarks' for the table of contents,
    % internal cross-reference links, web links for URLs, etc.)
    \usepackage{hyperref}
    \usepackage{longtable} % longtable support required by pandoc >1.10
    \usepackage{booktabs}  % table support for pandoc > 1.12.2
    \usepackage[normalem]{ulem} % ulem is needed to support strikethroughs (\sout)
                                % normalem makes italics be italics, not underlines
    

    
    
    % Colors for the hyperref package
    \definecolor{urlcolor}{rgb}{0,.145,.698}
    \definecolor{linkcolor}{rgb}{.71,0.21,0.01}
    \definecolor{citecolor}{rgb}{.12,.54,.11}

    % ANSI colors
    \definecolor{ansi-black}{HTML}{3E424D}
    \definecolor{ansi-black-intense}{HTML}{282C36}
    \definecolor{ansi-red}{HTML}{E75C58}
    \definecolor{ansi-red-intense}{HTML}{B22B31}
    \definecolor{ansi-green}{HTML}{00A250}
    \definecolor{ansi-green-intense}{HTML}{007427}
    \definecolor{ansi-yellow}{HTML}{DDB62B}
    \definecolor{ansi-yellow-intense}{HTML}{B27D12}
    \definecolor{ansi-blue}{HTML}{208FFB}
    \definecolor{ansi-blue-intense}{HTML}{0065CA}
    \definecolor{ansi-magenta}{HTML}{D160C4}
    \definecolor{ansi-magenta-intense}{HTML}{A03196}
    \definecolor{ansi-cyan}{HTML}{60C6C8}
    \definecolor{ansi-cyan-intense}{HTML}{258F8F}
    \definecolor{ansi-white}{HTML}{C5C1B4}
    \definecolor{ansi-white-intense}{HTML}{A1A6B2}

    % commands and environments needed by pandoc snippets
    % extracted from the output of `pandoc -s`
    \providecommand{\tightlist}{%
      \setlength{\itemsep}{0pt}\setlength{\parskip}{0pt}}
    \DefineVerbatimEnvironment{Highlighting}{Verbatim}{commandchars=\\\{\}}
    % Add ',fontsize=\small' for more characters per line
    \newenvironment{Shaded}{}{}
    \newcommand{\KeywordTok}[1]{\textcolor[rgb]{0.00,0.44,0.13}{\textbf{{#1}}}}
    \newcommand{\DataTypeTok}[1]{\textcolor[rgb]{0.56,0.13,0.00}{{#1}}}
    \newcommand{\DecValTok}[1]{\textcolor[rgb]{0.25,0.63,0.44}{{#1}}}
    \newcommand{\BaseNTok}[1]{\textcolor[rgb]{0.25,0.63,0.44}{{#1}}}
    \newcommand{\FloatTok}[1]{\textcolor[rgb]{0.25,0.63,0.44}{{#1}}}
    \newcommand{\CharTok}[1]{\textcolor[rgb]{0.25,0.44,0.63}{{#1}}}
    \newcommand{\StringTok}[1]{\textcolor[rgb]{0.25,0.44,0.63}{{#1}}}
    \newcommand{\CommentTok}[1]{\textcolor[rgb]{0.38,0.63,0.69}{\textit{{#1}}}}
    \newcommand{\OtherTok}[1]{\textcolor[rgb]{0.00,0.44,0.13}{{#1}}}
    \newcommand{\AlertTok}[1]{\textcolor[rgb]{1.00,0.00,0.00}{\textbf{{#1}}}}
    \newcommand{\FunctionTok}[1]{\textcolor[rgb]{0.02,0.16,0.49}{{#1}}}
    \newcommand{\RegionMarkerTok}[1]{{#1}}
    \newcommand{\ErrorTok}[1]{\textcolor[rgb]{1.00,0.00,0.00}{\textbf{{#1}}}}
    \newcommand{\NormalTok}[1]{{#1}}
    
    % Additional commands for more recent versions of Pandoc
    \newcommand{\ConstantTok}[1]{\textcolor[rgb]{0.53,0.00,0.00}{{#1}}}
    \newcommand{\SpecialCharTok}[1]{\textcolor[rgb]{0.25,0.44,0.63}{{#1}}}
    \newcommand{\VerbatimStringTok}[1]{\textcolor[rgb]{0.25,0.44,0.63}{{#1}}}
    \newcommand{\SpecialStringTok}[1]{\textcolor[rgb]{0.73,0.40,0.53}{{#1}}}
    \newcommand{\ImportTok}[1]{{#1}}
    \newcommand{\DocumentationTok}[1]{\textcolor[rgb]{0.73,0.13,0.13}{\textit{{#1}}}}
    \newcommand{\AnnotationTok}[1]{\textcolor[rgb]{0.38,0.63,0.69}{\textbf{\textit{{#1}}}}}
    \newcommand{\CommentVarTok}[1]{\textcolor[rgb]{0.38,0.63,0.69}{\textbf{\textit{{#1}}}}}
    \newcommand{\VariableTok}[1]{\textcolor[rgb]{0.10,0.09,0.49}{{#1}}}
    \newcommand{\ControlFlowTok}[1]{\textcolor[rgb]{0.00,0.44,0.13}{\textbf{{#1}}}}
    \newcommand{\OperatorTok}[1]{\textcolor[rgb]{0.40,0.40,0.40}{{#1}}}
    \newcommand{\BuiltInTok}[1]{{#1}}
    \newcommand{\ExtensionTok}[1]{{#1}}
    \newcommand{\PreprocessorTok}[1]{\textcolor[rgb]{0.74,0.48,0.00}{{#1}}}
    \newcommand{\AttributeTok}[1]{\textcolor[rgb]{0.49,0.56,0.16}{{#1}}}
    \newcommand{\InformationTok}[1]{\textcolor[rgb]{0.38,0.63,0.69}{\textbf{\textit{{#1}}}}}
    \newcommand{\WarningTok}[1]{\textcolor[rgb]{0.38,0.63,0.69}{\textbf{\textit{{#1}}}}}
    
    
    % Define a nice break command that doesn't care if a line doesn't already
    % exist.
    \def\br{\hspace*{\fill} \\* }
    % Math Jax compatability definitions
    \def\gt{>}
    \def\lt{<}
    % Document parameters
    \title{04-Semantics-Operators}
    
    
    

    % Pygments definitions
    
\makeatletter
\def\PY@reset{\let\PY@it=\relax \let\PY@bf=\relax%
    \let\PY@ul=\relax \let\PY@tc=\relax%
    \let\PY@bc=\relax \let\PY@ff=\relax}
\def\PY@tok#1{\csname PY@tok@#1\endcsname}
\def\PY@toks#1+{\ifx\relax#1\empty\else%
    \PY@tok{#1}\expandafter\PY@toks\fi}
\def\PY@do#1{\PY@bc{\PY@tc{\PY@ul{%
    \PY@it{\PY@bf{\PY@ff{#1}}}}}}}
\def\PY#1#2{\PY@reset\PY@toks#1+\relax+\PY@do{#2}}

\expandafter\def\csname PY@tok@err\endcsname{\def\PY@bc##1{\setlength{\fboxsep}{0pt}\fcolorbox[rgb]{1.00,0.00,0.00}{1,1,1}{\strut ##1}}}
\expandafter\def\csname PY@tok@kp\endcsname{\def\PY@tc##1{\textcolor[rgb]{0.00,0.50,0.00}{##1}}}
\expandafter\def\csname PY@tok@bp\endcsname{\def\PY@tc##1{\textcolor[rgb]{0.00,0.50,0.00}{##1}}}
\expandafter\def\csname PY@tok@si\endcsname{\let\PY@bf=\textbf\def\PY@tc##1{\textcolor[rgb]{0.73,0.40,0.53}{##1}}}
\expandafter\def\csname PY@tok@sx\endcsname{\def\PY@tc##1{\textcolor[rgb]{0.00,0.50,0.00}{##1}}}
\expandafter\def\csname PY@tok@vc\endcsname{\def\PY@tc##1{\textcolor[rgb]{0.10,0.09,0.49}{##1}}}
\expandafter\def\csname PY@tok@cp\endcsname{\def\PY@tc##1{\textcolor[rgb]{0.74,0.48,0.00}{##1}}}
\expandafter\def\csname PY@tok@gi\endcsname{\def\PY@tc##1{\textcolor[rgb]{0.00,0.63,0.00}{##1}}}
\expandafter\def\csname PY@tok@nn\endcsname{\let\PY@bf=\textbf\def\PY@tc##1{\textcolor[rgb]{0.00,0.00,1.00}{##1}}}
\expandafter\def\csname PY@tok@vi\endcsname{\def\PY@tc##1{\textcolor[rgb]{0.10,0.09,0.49}{##1}}}
\expandafter\def\csname PY@tok@k\endcsname{\let\PY@bf=\textbf\def\PY@tc##1{\textcolor[rgb]{0.00,0.50,0.00}{##1}}}
\expandafter\def\csname PY@tok@mf\endcsname{\def\PY@tc##1{\textcolor[rgb]{0.40,0.40,0.40}{##1}}}
\expandafter\def\csname PY@tok@sh\endcsname{\def\PY@tc##1{\textcolor[rgb]{0.73,0.13,0.13}{##1}}}
\expandafter\def\csname PY@tok@cs\endcsname{\let\PY@it=\textit\def\PY@tc##1{\textcolor[rgb]{0.25,0.50,0.50}{##1}}}
\expandafter\def\csname PY@tok@nv\endcsname{\def\PY@tc##1{\textcolor[rgb]{0.10,0.09,0.49}{##1}}}
\expandafter\def\csname PY@tok@kn\endcsname{\let\PY@bf=\textbf\def\PY@tc##1{\textcolor[rgb]{0.00,0.50,0.00}{##1}}}
\expandafter\def\csname PY@tok@ow\endcsname{\let\PY@bf=\textbf\def\PY@tc##1{\textcolor[rgb]{0.67,0.13,1.00}{##1}}}
\expandafter\def\csname PY@tok@nc\endcsname{\let\PY@bf=\textbf\def\PY@tc##1{\textcolor[rgb]{0.00,0.00,1.00}{##1}}}
\expandafter\def\csname PY@tok@c1\endcsname{\let\PY@it=\textit\def\PY@tc##1{\textcolor[rgb]{0.25,0.50,0.50}{##1}}}
\expandafter\def\csname PY@tok@no\endcsname{\def\PY@tc##1{\textcolor[rgb]{0.53,0.00,0.00}{##1}}}
\expandafter\def\csname PY@tok@na\endcsname{\def\PY@tc##1{\textcolor[rgb]{0.49,0.56,0.16}{##1}}}
\expandafter\def\csname PY@tok@sc\endcsname{\def\PY@tc##1{\textcolor[rgb]{0.73,0.13,0.13}{##1}}}
\expandafter\def\csname PY@tok@ge\endcsname{\let\PY@it=\textit}
\expandafter\def\csname PY@tok@c\endcsname{\let\PY@it=\textit\def\PY@tc##1{\textcolor[rgb]{0.25,0.50,0.50}{##1}}}
\expandafter\def\csname PY@tok@s\endcsname{\def\PY@tc##1{\textcolor[rgb]{0.73,0.13,0.13}{##1}}}
\expandafter\def\csname PY@tok@sd\endcsname{\let\PY@it=\textit\def\PY@tc##1{\textcolor[rgb]{0.73,0.13,0.13}{##1}}}
\expandafter\def\csname PY@tok@nf\endcsname{\def\PY@tc##1{\textcolor[rgb]{0.00,0.00,1.00}{##1}}}
\expandafter\def\csname PY@tok@nt\endcsname{\let\PY@bf=\textbf\def\PY@tc##1{\textcolor[rgb]{0.00,0.50,0.00}{##1}}}
\expandafter\def\csname PY@tok@nb\endcsname{\def\PY@tc##1{\textcolor[rgb]{0.00,0.50,0.00}{##1}}}
\expandafter\def\csname PY@tok@gu\endcsname{\let\PY@bf=\textbf\def\PY@tc##1{\textcolor[rgb]{0.50,0.00,0.50}{##1}}}
\expandafter\def\csname PY@tok@s2\endcsname{\def\PY@tc##1{\textcolor[rgb]{0.73,0.13,0.13}{##1}}}
\expandafter\def\csname PY@tok@w\endcsname{\def\PY@tc##1{\textcolor[rgb]{0.73,0.73,0.73}{##1}}}
\expandafter\def\csname PY@tok@mo\endcsname{\def\PY@tc##1{\textcolor[rgb]{0.40,0.40,0.40}{##1}}}
\expandafter\def\csname PY@tok@o\endcsname{\def\PY@tc##1{\textcolor[rgb]{0.40,0.40,0.40}{##1}}}
\expandafter\def\csname PY@tok@cpf\endcsname{\let\PY@it=\textit\def\PY@tc##1{\textcolor[rgb]{0.25,0.50,0.50}{##1}}}
\expandafter\def\csname PY@tok@kc\endcsname{\let\PY@bf=\textbf\def\PY@tc##1{\textcolor[rgb]{0.00,0.50,0.00}{##1}}}
\expandafter\def\csname PY@tok@sb\endcsname{\def\PY@tc##1{\textcolor[rgb]{0.73,0.13,0.13}{##1}}}
\expandafter\def\csname PY@tok@gs\endcsname{\let\PY@bf=\textbf}
\expandafter\def\csname PY@tok@vg\endcsname{\def\PY@tc##1{\textcolor[rgb]{0.10,0.09,0.49}{##1}}}
\expandafter\def\csname PY@tok@ch\endcsname{\let\PY@it=\textit\def\PY@tc##1{\textcolor[rgb]{0.25,0.50,0.50}{##1}}}
\expandafter\def\csname PY@tok@gd\endcsname{\def\PY@tc##1{\textcolor[rgb]{0.63,0.00,0.00}{##1}}}
\expandafter\def\csname PY@tok@kr\endcsname{\let\PY@bf=\textbf\def\PY@tc##1{\textcolor[rgb]{0.00,0.50,0.00}{##1}}}
\expandafter\def\csname PY@tok@mi\endcsname{\def\PY@tc##1{\textcolor[rgb]{0.40,0.40,0.40}{##1}}}
\expandafter\def\csname PY@tok@mb\endcsname{\def\PY@tc##1{\textcolor[rgb]{0.40,0.40,0.40}{##1}}}
\expandafter\def\csname PY@tok@gp\endcsname{\let\PY@bf=\textbf\def\PY@tc##1{\textcolor[rgb]{0.00,0.00,0.50}{##1}}}
\expandafter\def\csname PY@tok@kd\endcsname{\let\PY@bf=\textbf\def\PY@tc##1{\textcolor[rgb]{0.00,0.50,0.00}{##1}}}
\expandafter\def\csname PY@tok@m\endcsname{\def\PY@tc##1{\textcolor[rgb]{0.40,0.40,0.40}{##1}}}
\expandafter\def\csname PY@tok@nd\endcsname{\def\PY@tc##1{\textcolor[rgb]{0.67,0.13,1.00}{##1}}}
\expandafter\def\csname PY@tok@cm\endcsname{\let\PY@it=\textit\def\PY@tc##1{\textcolor[rgb]{0.25,0.50,0.50}{##1}}}
\expandafter\def\csname PY@tok@gh\endcsname{\let\PY@bf=\textbf\def\PY@tc##1{\textcolor[rgb]{0.00,0.00,0.50}{##1}}}
\expandafter\def\csname PY@tok@ne\endcsname{\let\PY@bf=\textbf\def\PY@tc##1{\textcolor[rgb]{0.82,0.25,0.23}{##1}}}
\expandafter\def\csname PY@tok@kt\endcsname{\def\PY@tc##1{\textcolor[rgb]{0.69,0.00,0.25}{##1}}}
\expandafter\def\csname PY@tok@se\endcsname{\let\PY@bf=\textbf\def\PY@tc##1{\textcolor[rgb]{0.73,0.40,0.13}{##1}}}
\expandafter\def\csname PY@tok@sr\endcsname{\def\PY@tc##1{\textcolor[rgb]{0.73,0.40,0.53}{##1}}}
\expandafter\def\csname PY@tok@nl\endcsname{\def\PY@tc##1{\textcolor[rgb]{0.63,0.63,0.00}{##1}}}
\expandafter\def\csname PY@tok@gt\endcsname{\def\PY@tc##1{\textcolor[rgb]{0.00,0.27,0.87}{##1}}}
\expandafter\def\csname PY@tok@il\endcsname{\def\PY@tc##1{\textcolor[rgb]{0.40,0.40,0.40}{##1}}}
\expandafter\def\csname PY@tok@ss\endcsname{\def\PY@tc##1{\textcolor[rgb]{0.10,0.09,0.49}{##1}}}
\expandafter\def\csname PY@tok@mh\endcsname{\def\PY@tc##1{\textcolor[rgb]{0.40,0.40,0.40}{##1}}}
\expandafter\def\csname PY@tok@go\endcsname{\def\PY@tc##1{\textcolor[rgb]{0.53,0.53,0.53}{##1}}}
\expandafter\def\csname PY@tok@s1\endcsname{\def\PY@tc##1{\textcolor[rgb]{0.73,0.13,0.13}{##1}}}
\expandafter\def\csname PY@tok@gr\endcsname{\def\PY@tc##1{\textcolor[rgb]{1.00,0.00,0.00}{##1}}}
\expandafter\def\csname PY@tok@ni\endcsname{\let\PY@bf=\textbf\def\PY@tc##1{\textcolor[rgb]{0.60,0.60,0.60}{##1}}}

\def\PYZbs{\char`\\}
\def\PYZus{\char`\_}
\def\PYZob{\char`\{}
\def\PYZcb{\char`\}}
\def\PYZca{\char`\^}
\def\PYZam{\char`\&}
\def\PYZlt{\char`\<}
\def\PYZgt{\char`\>}
\def\PYZsh{\char`\#}
\def\PYZpc{\char`\%}
\def\PYZdl{\char`\$}
\def\PYZhy{\char`\-}
\def\PYZsq{\char`\'}
\def\PYZdq{\char`\"}
\def\PYZti{\char`\~}
% for compatibility with earlier versions
\def\PYZat{@}
\def\PYZlb{[}
\def\PYZrb{]}
\makeatother


    % Exact colors from NB
    \definecolor{incolor}{rgb}{0.0, 0.0, 0.5}
    \definecolor{outcolor}{rgb}{0.545, 0.0, 0.0}



    
    % Prevent overflowing lines due to hard-to-break entities
    \sloppy 
    % Setup hyperref package
    \hypersetup{
      breaklinks=true,  % so long urls are correctly broken across lines
      colorlinks=true,
      urlcolor=urlcolor,
      linkcolor=linkcolor,
      citecolor=citecolor,
      }
    % Slightly bigger margins than the latex defaults
    
    \geometry{verbose,tmargin=1in,bmargin=1in,lmargin=1in,rmargin=1in}
    
    

    \begin{document}
    
    
    \maketitle
    
    

    
    \section{Basic Python Semantics:
Operators}\label{basic-python-semantics-operators}

    In the previous section, we began to look at the semantics of Python
variables and objects; here we'll dig into the semantics of the various
\emph{operators} included in the language. By the end of this section,
you'll have the basic tools to begin comparing and operating on data in
Python.

    \subsection{Arithmetic Operations}\label{arithmetic-operations}

Python implements seven basic binary arithmetic operators, two of which
can double as unary operators. They are summarized in the following
table:

\begin{longtable}[c]{@{}lll@{}}
\toprule
Operator & Name & Description\tabularnewline
\midrule
\endhead
\texttt{a\ +\ b} & Addition & Sum of \texttt{a} and
\texttt{b}\tabularnewline
\texttt{a\ -\ b} & Subtraction & Difference of \texttt{a} and
\texttt{b}\tabularnewline
\texttt{a\ *\ b} & Multiplication & Product of \texttt{a} and
\texttt{b}\tabularnewline
\texttt{a\ /\ b} & True division & Quotient of \texttt{a} and
\texttt{b}\tabularnewline
\texttt{a\ //\ b} & Floor division & Quotient of \texttt{a} and
\texttt{b}, removing fractional parts\tabularnewline
\texttt{a\ \%\ b} & Modulus & Integer remainder after division of
\texttt{a} by \texttt{b}\tabularnewline
\texttt{a\ **\ b} & Exponentiation & \texttt{a} raised to the power of
\texttt{b}\tabularnewline
\texttt{-a} & Negation & The negative of \texttt{a}\tabularnewline
\texttt{+a} & Unary plus & \texttt{a} unchanged (rarely
used)\tabularnewline
\bottomrule
\end{longtable}

These operators can be used and combined in intuitive ways, using
standard parentheses to group operations. For example:

    \begin{Verbatim}[commandchars=\\\{\}]
{\color{incolor}In [{\color{incolor}1}]:} \PY{c+c1}{\PYZsh{} addition, subtraction, multiplication}
        \PY{p}{(}\PY{l+m+mi}{4} \PY{o}{+} \PY{l+m+mi}{8}\PY{p}{)} \PY{o}{*} \PY{p}{(}\PY{l+m+mf}{6.5} \PY{o}{\PYZhy{}} \PY{l+m+mi}{3}\PY{p}{)}
\end{Verbatim}

            \begin{Verbatim}[commandchars=\\\{\}]
{\color{outcolor}Out[{\color{outcolor}1}]:} 42.0
\end{Verbatim}
        
    Floor division is true division with fractional parts truncated:

    \begin{Verbatim}[commandchars=\\\{\}]
{\color{incolor}In [{\color{incolor}2}]:} \PY{c+c1}{\PYZsh{} True division}
        \PY{n+nb}{print}\PY{p}{(}\PY{l+m+mi}{11} \PY{o}{/} \PY{l+m+mi}{2}\PY{p}{)}
\end{Verbatim}

    \begin{Verbatim}[commandchars=\\\{\}]
5.5

    \end{Verbatim}

    \begin{Verbatim}[commandchars=\\\{\}]
{\color{incolor}In [{\color{incolor}3}]:} \PY{c+c1}{\PYZsh{} Floor division}
        \PY{n+nb}{print}\PY{p}{(}\PY{l+m+mi}{11} \PY{o}{/}\PY{o}{/} \PY{l+m+mi}{2}\PY{p}{)}
\end{Verbatim}

    \begin{Verbatim}[commandchars=\\\{\}]
5

    \end{Verbatim}

    The floor division operator was added in Python 3; you should be aware
if working in Python 2 that the standard division operator (\texttt{/})
acts like floor division for integers and like true division for
floating-point numbers.

Finally, I'll mention an eighth arithmetic operator that was added in
Python 3.5: the \texttt{a\ @\ b} operator, which is meant to indicate
the \emph{matrix product} of \texttt{a} and \texttt{b}, for use in
various linear algebra packages.

    \subsection{Bitwise Operations}\label{bitwise-operations}

In addition to the standard numerical operations, Python includes
operators to perform bitwise logical operations on integers. These are
much less commonly used than the standard arithmetic operations, but
it's useful to know that they exist. The six bitwise operators are
summarized in the following table:

\begin{longtable}[c]{@{}lll@{}}
\toprule
Operator & Name & Description\tabularnewline
\midrule
\endhead
\texttt{a\ \&\ b} & Bitwise AND & Bits defined in both \texttt{a} and
\texttt{b}\tabularnewline
a \textbar{} b & Bitwise OR & Bits defined in \texttt{a} or \texttt{b}
or both\tabularnewline
\texttt{a\ \^{}\ b} & Bitwise XOR & Bits defined in \texttt{a} or
\texttt{b} but not both\tabularnewline
\texttt{a\ \textless{}\textless{}\ b} & Bit shift left & Shift bits of
\texttt{a} left by \texttt{b} units\tabularnewline
\texttt{a\ \textgreater{}\textgreater{}\ b} & Bit shift right & Shift
bits of \texttt{a} right by \texttt{b} units\tabularnewline
\texttt{\textasciitilde{}a} & Bitwise NOT & Bitwise negation of
\texttt{a}\tabularnewline
\bottomrule
\end{longtable}

These bitwise operators only make sense in terms of the binary
representation of numbers, which you can see using the built-in
\texttt{bin} function:

    \begin{Verbatim}[commandchars=\\\{\}]
{\color{incolor}In [{\color{incolor}4}]:} \PY{n+nb}{bin}\PY{p}{(}\PY{l+m+mi}{10}\PY{p}{)}
\end{Verbatim}

            \begin{Verbatim}[commandchars=\\\{\}]
{\color{outcolor}Out[{\color{outcolor}4}]:} '0b1010'
\end{Verbatim}
        
    The result is prefixed with
\texttt{\textquotesingle{}0b\textquotesingle{}}, which indicates a
binary representation. The rest of the digits indicate that the number
10 is expressed as the sum
\(1 \cdot 2^3 + 0 \cdot 2^2 + 1 \cdot 2^1 + 0 \cdot 2^0\). Similarly, we
can write:

    \begin{Verbatim}[commandchars=\\\{\}]
{\color{incolor}In [{\color{incolor}5}]:} \PY{n+nb}{bin}\PY{p}{(}\PY{l+m+mi}{4}\PY{p}{)}
\end{Verbatim}

            \begin{Verbatim}[commandchars=\\\{\}]
{\color{outcolor}Out[{\color{outcolor}5}]:} '0b100'
\end{Verbatim}
        
    Now, using bitwise OR, we can find the number which combines the bits of
4 and 10:

    \begin{Verbatim}[commandchars=\\\{\}]
{\color{incolor}In [{\color{incolor}6}]:} \PY{l+m+mi}{4} \PY{o}{|} \PY{l+m+mi}{10}
\end{Verbatim}

            \begin{Verbatim}[commandchars=\\\{\}]
{\color{outcolor}Out[{\color{outcolor}6}]:} 14
\end{Verbatim}
        
    \begin{Verbatim}[commandchars=\\\{\}]
{\color{incolor}In [{\color{incolor}7}]:} \PY{n+nb}{bin}\PY{p}{(}\PY{l+m+mi}{4} \PY{o}{|} \PY{l+m+mi}{10}\PY{p}{)}
\end{Verbatim}

            \begin{Verbatim}[commandchars=\\\{\}]
{\color{outcolor}Out[{\color{outcolor}7}]:} '0b1110'
\end{Verbatim}
        
    These bitwise operators are not as immediately useful as the standard
arithmetic operators, but it's helpful to see them at least once to
understand what class of operation they perform. In particular, users
from other languages are sometimes tempted to use XOR (i.e.,
\texttt{a\ \^{}\ b}) when they really mean exponentiation (i.e.,
\texttt{a\ **\ b}).

    \subsection{Assignment Operations}\label{assignment-operations}

We've seen that variables can be assigned with the ``\texttt{=}''
operator, and the values stored for later use. For example:

    \begin{Verbatim}[commandchars=\\\{\}]
{\color{incolor}In [{\color{incolor}8}]:} \PY{n}{a} \PY{o}{=} \PY{l+m+mi}{24}
        \PY{n+nb}{print}\PY{p}{(}\PY{n}{a}\PY{p}{)}
\end{Verbatim}

    \begin{Verbatim}[commandchars=\\\{\}]
24

    \end{Verbatim}

    We can use these variables in expressions with any of the operators
mentioned earlier. For example, to add 2 to \texttt{a} we write:

    \begin{Verbatim}[commandchars=\\\{\}]
{\color{incolor}In [{\color{incolor}9}]:} \PY{n}{a} \PY{o}{+} \PY{l+m+mi}{2}
\end{Verbatim}

            \begin{Verbatim}[commandchars=\\\{\}]
{\color{outcolor}Out[{\color{outcolor}9}]:} 26
\end{Verbatim}
        
    We might want to update the variable \texttt{a} with this new value; in
this case, we could combine the addition and the assignment and write
\texttt{a\ =\ a\ +\ 2}. Because this type of combined operation and
assignment is so common, Python includes built-in update operators for
all of the arithmetic operations:

    \begin{Verbatim}[commandchars=\\\{\}]
{\color{incolor}In [{\color{incolor}10}]:} \PY{n}{a} \PY{o}{+}\PY{o}{=} \PY{l+m+mi}{2}  \PY{c+c1}{\PYZsh{} equivalent to a = a + 2}
         \PY{n+nb}{print}\PY{p}{(}\PY{n}{a}\PY{p}{)}
\end{Verbatim}

    \begin{Verbatim}[commandchars=\\\{\}]
26

    \end{Verbatim}

    There is an augmented assignment operator corresponding to each of the
binary operators listed earlier; in brief, they are:

\begin{longtable}[c]{@{}ll@{}}
\toprule
\texttt{a\ +=\ b} & \texttt{a\ -=\ b}\tabularnewline
\texttt{a\ //=\ b} & \texttt{a\ \%=\ b}\tabularnewline
a \textbar{}= b & \texttt{a\ \^{}=\ b}\tabularnewline
\bottomrule
\end{longtable}

Each one is equivalent to the corresponding operation followed by
assignment: that is, for any operator ``\texttt{?}'', the expression
\texttt{a\ ?=\ b} is equivalent to \texttt{a\ =\ a\ ?\ b}, with a slight
catch. For mutable objects like lists, arrays, or DataFrames, these
augmented assignment operations are actually subtly different than their
more verbose counterparts: they modify the contents of the original
object rather than creating a new object to store the result.

    \subsection{Comparison Operations}\label{comparison-operations}

Another type of operation which can be very useful is comparison of
different values. For this, Python implements standard comparison
operators, which return Boolean values \texttt{True} and \texttt{False}.
The comparison operations are listed in the following table:

\textbar{} Operation \textbar{} Description \textbar{}\textbar{}
Operation \textbar{} Description \textbar{}
\textbar{}---------------\textbar{}-----------------------------------\textbar{}\textbar{}---------------\textbar{}--------------------------------------\textbar{}
\textbar{} \texttt{a\ ==\ b} \textbar{} \texttt{a} equal to \texttt{b}
\textbar{}\textbar{} \texttt{a\ !=\ b} \textbar{} \texttt{a} not equal
to \texttt{b} \textbar{} \textbar{} \texttt{a\ \textless{}\ b}
\textbar{} \texttt{a} less than \texttt{b} \textbar{}\textbar{}
\texttt{a\ \textgreater{}\ b} \textbar{} \texttt{a} greater than
\texttt{b} \textbar{} \textbar{} \texttt{a\ \textless{}=\ b} \textbar{}
\texttt{a} less than or equal to \texttt{b} \textbar{}\textbar{}
\texttt{a\ \textgreater{}=\ b} \textbar{} \texttt{a} greater than or
equal to \texttt{b} \textbar{}

These comparison operators can be combined with the arithmetic and
bitwise operators to express a virtually limitless range of tests for
the numbers. For example, we can check if a number is odd by checking
that the modulus with 2 returns 1:

    \begin{Verbatim}[commandchars=\\\{\}]
{\color{incolor}In [{\color{incolor}11}]:} \PY{c+c1}{\PYZsh{} 25 is odd}
         \PY{l+m+mi}{25} \PY{o}{\PYZpc{}} \PY{l+m+mi}{2} \PY{o}{==} \PY{l+m+mi}{1}
\end{Verbatim}

            \begin{Verbatim}[commandchars=\\\{\}]
{\color{outcolor}Out[{\color{outcolor}11}]:} True
\end{Verbatim}
        
    \begin{Verbatim}[commandchars=\\\{\}]
{\color{incolor}In [{\color{incolor}12}]:} \PY{c+c1}{\PYZsh{} 66 is odd}
         \PY{l+m+mi}{66} \PY{o}{\PYZpc{}} \PY{l+m+mi}{2} \PY{o}{==} \PY{l+m+mi}{1}
\end{Verbatim}

            \begin{Verbatim}[commandchars=\\\{\}]
{\color{outcolor}Out[{\color{outcolor}12}]:} False
\end{Verbatim}
        
    We can string-together multiple comparisons to check more complicated
relationships:

    \begin{Verbatim}[commandchars=\\\{\}]
{\color{incolor}In [{\color{incolor}13}]:} \PY{c+c1}{\PYZsh{} check if a is between 15 and 30}
         \PY{n}{a} \PY{o}{=} \PY{l+m+mi}{25}
         \PY{l+m+mi}{15} \PY{o}{\PYZlt{}} \PY{n}{a} \PY{o}{\PYZlt{}} \PY{l+m+mi}{30}
\end{Verbatim}

            \begin{Verbatim}[commandchars=\\\{\}]
{\color{outcolor}Out[{\color{outcolor}13}]:} True
\end{Verbatim}
        
    And, just to make your head hurt a bit, take a look at this comparison:

    \begin{Verbatim}[commandchars=\\\{\}]
{\color{incolor}In [{\color{incolor}14}]:} \PY{o}{\PYZhy{}}\PY{l+m+mi}{1} \PY{o}{==} \PY{o}{\PYZti{}}\PY{l+m+mi}{0}
\end{Verbatim}

            \begin{Verbatim}[commandchars=\\\{\}]
{\color{outcolor}Out[{\color{outcolor}14}]:} True
\end{Verbatim}
        
    Recall that \texttt{\textasciitilde{}} is the bit-flip operator, and
evidently when you flip all the bits of zero you end up with -1. If
you're curious as to why this is, look up the \emph{two's complement}
integer encoding scheme, which is what Python uses to encode signed
integers, and think about happens when you start flipping all the bits
of integers encoded this way.

    \subsection{Boolean Operations}\label{boolean-operations}

When working with Boolean values, Python provides operators to combine
the values using the standard concepts of ``and'', ``or'', and ``not''.
Predictably, these operators are expressed using the words \texttt{and},
\texttt{or}, and \texttt{not}:

    \begin{Verbatim}[commandchars=\\\{\}]
{\color{incolor}In [{\color{incolor}15}]:} \PY{n}{x} \PY{o}{=} \PY{l+m+mi}{4}
         \PY{p}{(}\PY{n}{x} \PY{o}{\PYZlt{}} \PY{l+m+mi}{6}\PY{p}{)} \PY{o+ow}{and} \PY{p}{(}\PY{n}{x} \PY{o}{\PYZgt{}} \PY{l+m+mi}{2}\PY{p}{)}
\end{Verbatim}

            \begin{Verbatim}[commandchars=\\\{\}]
{\color{outcolor}Out[{\color{outcolor}15}]:} True
\end{Verbatim}
        
    \begin{Verbatim}[commandchars=\\\{\}]
{\color{incolor}In [{\color{incolor}16}]:} \PY{p}{(}\PY{n}{x} \PY{o}{\PYZgt{}} \PY{l+m+mi}{10}\PY{p}{)} \PY{o+ow}{or} \PY{p}{(}\PY{n}{x} \PY{o}{\PYZpc{}} \PY{l+m+mi}{2} \PY{o}{==} \PY{l+m+mi}{0}\PY{p}{)}
\end{Verbatim}

            \begin{Verbatim}[commandchars=\\\{\}]
{\color{outcolor}Out[{\color{outcolor}16}]:} True
\end{Verbatim}
        
    \begin{Verbatim}[commandchars=\\\{\}]
{\color{incolor}In [{\color{incolor}17}]:} \PY{o+ow}{not} \PY{p}{(}\PY{n}{x} \PY{o}{\PYZlt{}} \PY{l+m+mi}{6}\PY{p}{)}
\end{Verbatim}

            \begin{Verbatim}[commandchars=\\\{\}]
{\color{outcolor}Out[{\color{outcolor}17}]:} False
\end{Verbatim}
        
    Boolean algebra aficionados might notice that the XOR operator is not
included; this can of course be constructed in several ways from a
compound statement of the other operators. Otherwise, a clever trick you
can use for XOR of Boolean values is the following:

    \begin{Verbatim}[commandchars=\\\{\}]
{\color{incolor}In [{\color{incolor}18}]:} \PY{c+c1}{\PYZsh{} (x \PYZgt{} 1) xor (x \PYZlt{} 10)}
         \PY{p}{(}\PY{n}{x} \PY{o}{\PYZgt{}} \PY{l+m+mi}{1}\PY{p}{)} \PY{o}{!=} \PY{p}{(}\PY{n}{x} \PY{o}{\PYZlt{}} \PY{l+m+mi}{10}\PY{p}{)}
\end{Verbatim}

            \begin{Verbatim}[commandchars=\\\{\}]
{\color{outcolor}Out[{\color{outcolor}18}]:} False
\end{Verbatim}
        
    These sorts of Boolean operations will become extremely useful when we
begin discussing \emph{control flow statements} such as conditionals and
loops.

One sometimes confusing thing about the language is when to use Boolean
operators (\texttt{and}, \texttt{or}, \texttt{not}), and when to use
bitwise operations (\texttt{\&}, \texttt{\textbar{}},
\texttt{\textasciitilde{}}). The answer lies in their names: Boolean
operators should be used when you want to compute \emph{Boolean values
(i.e., truth or falsehood) of entire statements}. Bitwise operations
should be used when you want to \emph{operate on individual bits or
components of the objects in question}.

    \subsection{Identity and Membership
Operators}\label{identity-and-membership-operators}

Like \texttt{and}, \texttt{or}, and \texttt{not}, Python also contains
prose-like operators to check for identity and membership. They are the
following:

\begin{longtable}[c]{@{}ll@{}}
\toprule
Operator & Description\tabularnewline
\midrule
\endhead
\texttt{a\ is\ b} & True if \texttt{a} and \texttt{b} are identical
objects\tabularnewline
\texttt{a\ is\ not\ b} & True if \texttt{a} and \texttt{b} are not
identical objects\tabularnewline
\texttt{a\ in\ b} & True if \texttt{a} is a member of
\texttt{b}\tabularnewline
\texttt{a\ not\ in\ b} & True if \texttt{a} is not a member of
\texttt{b}\tabularnewline
\bottomrule
\end{longtable}

    \subsubsection{\texorpdfstring{Identity Operators: ``\texttt{is}'' and
``\texttt{is\ not}''}{Identity Operators: is and is not}}\label{identity-operators-is-and-is-not}

The identity operators, ``\texttt{is}'' and ``\texttt{is\ not}'' check
for \emph{object identity}. Object identity is different than equality,
as we can see here:

    \begin{Verbatim}[commandchars=\\\{\}]
{\color{incolor}In [{\color{incolor}19}]:} \PY{n}{a} \PY{o}{=} \PY{p}{[}\PY{l+m+mi}{1}\PY{p}{,} \PY{l+m+mi}{2}\PY{p}{,} \PY{l+m+mi}{3}\PY{p}{]}
         \PY{n}{b} \PY{o}{=} \PY{p}{[}\PY{l+m+mi}{1}\PY{p}{,} \PY{l+m+mi}{2}\PY{p}{,} \PY{l+m+mi}{3}\PY{p}{]}
\end{Verbatim}

    \begin{Verbatim}[commandchars=\\\{\}]
{\color{incolor}In [{\color{incolor}20}]:} \PY{n}{a} \PY{o}{==} \PY{n}{b}
\end{Verbatim}

            \begin{Verbatim}[commandchars=\\\{\}]
{\color{outcolor}Out[{\color{outcolor}20}]:} True
\end{Verbatim}
        
    \begin{Verbatim}[commandchars=\\\{\}]
{\color{incolor}In [{\color{incolor}21}]:} \PY{n}{a} \PY{o+ow}{is} \PY{n}{b}
\end{Verbatim}

            \begin{Verbatim}[commandchars=\\\{\}]
{\color{outcolor}Out[{\color{outcolor}21}]:} False
\end{Verbatim}
        
    \begin{Verbatim}[commandchars=\\\{\}]
{\color{incolor}In [{\color{incolor}22}]:} \PY{n}{a} \PY{o+ow}{is} \PY{o+ow}{not} \PY{n}{b}
\end{Verbatim}

            \begin{Verbatim}[commandchars=\\\{\}]
{\color{outcolor}Out[{\color{outcolor}22}]:} True
\end{Verbatim}
        
    What do identical objects look like? Here is an example:

    \begin{Verbatim}[commandchars=\\\{\}]
{\color{incolor}In [{\color{incolor}23}]:} \PY{n}{a} \PY{o}{=} \PY{p}{[}\PY{l+m+mi}{1}\PY{p}{,} \PY{l+m+mi}{2}\PY{p}{,} \PY{l+m+mi}{3}\PY{p}{]}
         \PY{n}{b} \PY{o}{=} \PY{n}{a}
         \PY{n}{a} \PY{o+ow}{is} \PY{n}{b}
\end{Verbatim}

            \begin{Verbatim}[commandchars=\\\{\}]
{\color{outcolor}Out[{\color{outcolor}23}]:} True
\end{Verbatim}
        
    The difference between the two cases here is that in the first,
\texttt{a} and \texttt{b} point to \emph{different objects}, while in
the second they point to the \emph{same object}. As we saw in the
previous section, Python variables are pointers. The ``\texttt{is}''
operator checks whether the two variables are pointing to the same
container (object), rather than referring to what the container
contains. With this in mind, in most cases that a beginner is tempted to
use ``\texttt{is}'' what they really mean is \texttt{==}.

    \subsubsection{Membership operators}\label{membership-operators}

Membership operators check for membership within compound objects. So,
for example, we can write:

    \begin{Verbatim}[commandchars=\\\{\}]
{\color{incolor}In [{\color{incolor}24}]:} \PY{l+m+mi}{1} \PY{o+ow}{in} \PY{p}{[}\PY{l+m+mi}{1}\PY{p}{,} \PY{l+m+mi}{2}\PY{p}{,} \PY{l+m+mi}{3}\PY{p}{]}
\end{Verbatim}

            \begin{Verbatim}[commandchars=\\\{\}]
{\color{outcolor}Out[{\color{outcolor}24}]:} True
\end{Verbatim}
        
    \begin{Verbatim}[commandchars=\\\{\}]
{\color{incolor}In [{\color{incolor}25}]:} \PY{l+m+mi}{2} \PY{o+ow}{not} \PY{o+ow}{in} \PY{p}{[}\PY{l+m+mi}{1}\PY{p}{,} \PY{l+m+mi}{2}\PY{p}{,} \PY{l+m+mi}{3}\PY{p}{]}
\end{Verbatim}

            \begin{Verbatim}[commandchars=\\\{\}]
{\color{outcolor}Out[{\color{outcolor}25}]:} False
\end{Verbatim}
        
    These membership operations are an example of what makes Python so easy
to use compared to lower-level languages such as C. In C, membership
would generally be determined by manually constructing a loop over the
list and checking for equality of each value. In Python, you just type
what you want to know, in a manner reminiscent of straightforward
English prose.


    % Add a bibliography block to the postdoc
    
    
    
    \end{document}
